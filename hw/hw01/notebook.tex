
% Default to the notebook output style

    


% Inherit from the specified cell style.




    
\documentclass[11pt]{article}

    
    
    \usepackage[T1]{fontenc}
    % Nicer default font (+ math font) than Computer Modern for most use cases
    \usepackage{mathpazo}

    % Basic figure setup, for now with no caption control since it's done
    % automatically by Pandoc (which extracts ![](path) syntax from Markdown).
    \usepackage{graphicx}
    % We will generate all images so they have a width \maxwidth. This means
    % that they will get their normal width if they fit onto the page, but
    % are scaled down if they would overflow the margins.
    \makeatletter
    \def\maxwidth{\ifdim\Gin@nat@width>\linewidth\linewidth
    \else\Gin@nat@width\fi}
    \makeatother
    \let\Oldincludegraphics\includegraphics
    % Set max figure width to be 80% of text width, for now hardcoded.
    \renewcommand{\includegraphics}[1]{\Oldincludegraphics[width=.8\maxwidth]{#1}}
    % Ensure that by default, figures have no caption (until we provide a
    % proper Figure object with a Caption API and a way to capture that
    % in the conversion process - todo).
    \usepackage{caption}
    \DeclareCaptionLabelFormat{nolabel}{}
    \captionsetup{labelformat=nolabel}

    \usepackage{adjustbox} % Used to constrain images to a maximum size 
    \usepackage{xcolor} % Allow colors to be defined
    \usepackage{enumerate} % Needed for markdown enumerations to work
    \usepackage{geometry} % Used to adjust the document margins
    \usepackage{amsmath} % Equations
    \usepackage{amssymb} % Equations
    \usepackage{textcomp} % defines textquotesingle
    % Hack from http://tex.stackexchange.com/a/47451/13684:
    \AtBeginDocument{%
        \def\PYZsq{\textquotesingle}% Upright quotes in Pygmentized code
    }
    \usepackage{upquote} % Upright quotes for verbatim code
    \usepackage{eurosym} % defines \euro
    \usepackage[mathletters]{ucs} % Extended unicode (utf-8) support
    \usepackage[utf8x]{inputenc} % Allow utf-8 characters in the tex document
    \usepackage{fancyvrb} % verbatim replacement that allows latex
    \usepackage{grffile} % extends the file name processing of package graphics 
                         % to support a larger range 
    % The hyperref package gives us a pdf with properly built
    % internal navigation ('pdf bookmarks' for the table of contents,
    % internal cross-reference links, web links for URLs, etc.)
    \usepackage{hyperref}
    \usepackage{longtable} % longtable support required by pandoc >1.10
    \usepackage{booktabs}  % table support for pandoc > 1.12.2
    \usepackage[inline]{enumitem} % IRkernel/repr support (it uses the enumerate* environment)
    \usepackage[normalem]{ulem} % ulem is needed to support strikethroughs (\sout)
                                % normalem makes italics be italics, not underlines
    

    
    
    % Colors for the hyperref package
    \definecolor{urlcolor}{rgb}{0,.145,.698}
    \definecolor{linkcolor}{rgb}{.71,0.21,0.01}
    \definecolor{citecolor}{rgb}{.12,.54,.11}

    % ANSI colors
    \definecolor{ansi-black}{HTML}{3E424D}
    \definecolor{ansi-black-intense}{HTML}{282C36}
    \definecolor{ansi-red}{HTML}{E75C58}
    \definecolor{ansi-red-intense}{HTML}{B22B31}
    \definecolor{ansi-green}{HTML}{00A250}
    \definecolor{ansi-green-intense}{HTML}{007427}
    \definecolor{ansi-yellow}{HTML}{DDB62B}
    \definecolor{ansi-yellow-intense}{HTML}{B27D12}
    \definecolor{ansi-blue}{HTML}{208FFB}
    \definecolor{ansi-blue-intense}{HTML}{0065CA}
    \definecolor{ansi-magenta}{HTML}{D160C4}
    \definecolor{ansi-magenta-intense}{HTML}{A03196}
    \definecolor{ansi-cyan}{HTML}{60C6C8}
    \definecolor{ansi-cyan-intense}{HTML}{258F8F}
    \definecolor{ansi-white}{HTML}{C5C1B4}
    \definecolor{ansi-white-intense}{HTML}{A1A6B2}

    % commands and environments needed by pandoc snippets
    % extracted from the output of `pandoc -s`
    \providecommand{\tightlist}{%
      \setlength{\itemsep}{0pt}\setlength{\parskip}{0pt}}
    \DefineVerbatimEnvironment{Highlighting}{Verbatim}{commandchars=\\\{\}}
    % Add ',fontsize=\small' for more characters per line
    \newenvironment{Shaded}{}{}
    \newcommand{\KeywordTok}[1]{\textcolor[rgb]{0.00,0.44,0.13}{\textbf{{#1}}}}
    \newcommand{\DataTypeTok}[1]{\textcolor[rgb]{0.56,0.13,0.00}{{#1}}}
    \newcommand{\DecValTok}[1]{\textcolor[rgb]{0.25,0.63,0.44}{{#1}}}
    \newcommand{\BaseNTok}[1]{\textcolor[rgb]{0.25,0.63,0.44}{{#1}}}
    \newcommand{\FloatTok}[1]{\textcolor[rgb]{0.25,0.63,0.44}{{#1}}}
    \newcommand{\CharTok}[1]{\textcolor[rgb]{0.25,0.44,0.63}{{#1}}}
    \newcommand{\StringTok}[1]{\textcolor[rgb]{0.25,0.44,0.63}{{#1}}}
    \newcommand{\CommentTok}[1]{\textcolor[rgb]{0.38,0.63,0.69}{\textit{{#1}}}}
    \newcommand{\OtherTok}[1]{\textcolor[rgb]{0.00,0.44,0.13}{{#1}}}
    \newcommand{\AlertTok}[1]{\textcolor[rgb]{1.00,0.00,0.00}{\textbf{{#1}}}}
    \newcommand{\FunctionTok}[1]{\textcolor[rgb]{0.02,0.16,0.49}{{#1}}}
    \newcommand{\RegionMarkerTok}[1]{{#1}}
    \newcommand{\ErrorTok}[1]{\textcolor[rgb]{1.00,0.00,0.00}{\textbf{{#1}}}}
    \newcommand{\NormalTok}[1]{{#1}}
    
    % Additional commands for more recent versions of Pandoc
    \newcommand{\ConstantTok}[1]{\textcolor[rgb]{0.53,0.00,0.00}{{#1}}}
    \newcommand{\SpecialCharTok}[1]{\textcolor[rgb]{0.25,0.44,0.63}{{#1}}}
    \newcommand{\VerbatimStringTok}[1]{\textcolor[rgb]{0.25,0.44,0.63}{{#1}}}
    \newcommand{\SpecialStringTok}[1]{\textcolor[rgb]{0.73,0.40,0.53}{{#1}}}
    \newcommand{\ImportTok}[1]{{#1}}
    \newcommand{\DocumentationTok}[1]{\textcolor[rgb]{0.73,0.13,0.13}{\textit{{#1}}}}
    \newcommand{\AnnotationTok}[1]{\textcolor[rgb]{0.38,0.63,0.69}{\textbf{\textit{{#1}}}}}
    \newcommand{\CommentVarTok}[1]{\textcolor[rgb]{0.38,0.63,0.69}{\textbf{\textit{{#1}}}}}
    \newcommand{\VariableTok}[1]{\textcolor[rgb]{0.10,0.09,0.49}{{#1}}}
    \newcommand{\ControlFlowTok}[1]{\textcolor[rgb]{0.00,0.44,0.13}{\textbf{{#1}}}}
    \newcommand{\OperatorTok}[1]{\textcolor[rgb]{0.40,0.40,0.40}{{#1}}}
    \newcommand{\BuiltInTok}[1]{{#1}}
    \newcommand{\ExtensionTok}[1]{{#1}}
    \newcommand{\PreprocessorTok}[1]{\textcolor[rgb]{0.74,0.48,0.00}{{#1}}}
    \newcommand{\AttributeTok}[1]{\textcolor[rgb]{0.49,0.56,0.16}{{#1}}}
    \newcommand{\InformationTok}[1]{\textcolor[rgb]{0.38,0.63,0.69}{\textbf{\textit{{#1}}}}}
    \newcommand{\WarningTok}[1]{\textcolor[rgb]{0.38,0.63,0.69}{\textbf{\textit{{#1}}}}}
    
    
    % Define a nice break command that doesn't care if a line doesn't already
    % exist.
    \def\br{\hspace*{\fill} \\* }
    % Math Jax compatability definitions
    \def\gt{>}
    \def\lt{<}
    % Document parameters
    \title{hw01}
    
    
    

    % Pygments definitions
    
\makeatletter
\def\PY@reset{\let\PY@it=\relax \let\PY@bf=\relax%
    \let\PY@ul=\relax \let\PY@tc=\relax%
    \let\PY@bc=\relax \let\PY@ff=\relax}
\def\PY@tok#1{\csname PY@tok@#1\endcsname}
\def\PY@toks#1+{\ifx\relax#1\empty\else%
    \PY@tok{#1}\expandafter\PY@toks\fi}
\def\PY@do#1{\PY@bc{\PY@tc{\PY@ul{%
    \PY@it{\PY@bf{\PY@ff{#1}}}}}}}
\def\PY#1#2{\PY@reset\PY@toks#1+\relax+\PY@do{#2}}

\expandafter\def\csname PY@tok@w\endcsname{\def\PY@tc##1{\textcolor[rgb]{0.73,0.73,0.73}{##1}}}
\expandafter\def\csname PY@tok@c\endcsname{\let\PY@it=\textit\def\PY@tc##1{\textcolor[rgb]{0.25,0.50,0.50}{##1}}}
\expandafter\def\csname PY@tok@cp\endcsname{\def\PY@tc##1{\textcolor[rgb]{0.74,0.48,0.00}{##1}}}
\expandafter\def\csname PY@tok@k\endcsname{\let\PY@bf=\textbf\def\PY@tc##1{\textcolor[rgb]{0.00,0.50,0.00}{##1}}}
\expandafter\def\csname PY@tok@kp\endcsname{\def\PY@tc##1{\textcolor[rgb]{0.00,0.50,0.00}{##1}}}
\expandafter\def\csname PY@tok@kt\endcsname{\def\PY@tc##1{\textcolor[rgb]{0.69,0.00,0.25}{##1}}}
\expandafter\def\csname PY@tok@o\endcsname{\def\PY@tc##1{\textcolor[rgb]{0.40,0.40,0.40}{##1}}}
\expandafter\def\csname PY@tok@ow\endcsname{\let\PY@bf=\textbf\def\PY@tc##1{\textcolor[rgb]{0.67,0.13,1.00}{##1}}}
\expandafter\def\csname PY@tok@nb\endcsname{\def\PY@tc##1{\textcolor[rgb]{0.00,0.50,0.00}{##1}}}
\expandafter\def\csname PY@tok@nf\endcsname{\def\PY@tc##1{\textcolor[rgb]{0.00,0.00,1.00}{##1}}}
\expandafter\def\csname PY@tok@nc\endcsname{\let\PY@bf=\textbf\def\PY@tc##1{\textcolor[rgb]{0.00,0.00,1.00}{##1}}}
\expandafter\def\csname PY@tok@nn\endcsname{\let\PY@bf=\textbf\def\PY@tc##1{\textcolor[rgb]{0.00,0.00,1.00}{##1}}}
\expandafter\def\csname PY@tok@ne\endcsname{\let\PY@bf=\textbf\def\PY@tc##1{\textcolor[rgb]{0.82,0.25,0.23}{##1}}}
\expandafter\def\csname PY@tok@nv\endcsname{\def\PY@tc##1{\textcolor[rgb]{0.10,0.09,0.49}{##1}}}
\expandafter\def\csname PY@tok@no\endcsname{\def\PY@tc##1{\textcolor[rgb]{0.53,0.00,0.00}{##1}}}
\expandafter\def\csname PY@tok@nl\endcsname{\def\PY@tc##1{\textcolor[rgb]{0.63,0.63,0.00}{##1}}}
\expandafter\def\csname PY@tok@ni\endcsname{\let\PY@bf=\textbf\def\PY@tc##1{\textcolor[rgb]{0.60,0.60,0.60}{##1}}}
\expandafter\def\csname PY@tok@na\endcsname{\def\PY@tc##1{\textcolor[rgb]{0.49,0.56,0.16}{##1}}}
\expandafter\def\csname PY@tok@nt\endcsname{\let\PY@bf=\textbf\def\PY@tc##1{\textcolor[rgb]{0.00,0.50,0.00}{##1}}}
\expandafter\def\csname PY@tok@nd\endcsname{\def\PY@tc##1{\textcolor[rgb]{0.67,0.13,1.00}{##1}}}
\expandafter\def\csname PY@tok@s\endcsname{\def\PY@tc##1{\textcolor[rgb]{0.73,0.13,0.13}{##1}}}
\expandafter\def\csname PY@tok@sd\endcsname{\let\PY@it=\textit\def\PY@tc##1{\textcolor[rgb]{0.73,0.13,0.13}{##1}}}
\expandafter\def\csname PY@tok@si\endcsname{\let\PY@bf=\textbf\def\PY@tc##1{\textcolor[rgb]{0.73,0.40,0.53}{##1}}}
\expandafter\def\csname PY@tok@se\endcsname{\let\PY@bf=\textbf\def\PY@tc##1{\textcolor[rgb]{0.73,0.40,0.13}{##1}}}
\expandafter\def\csname PY@tok@sr\endcsname{\def\PY@tc##1{\textcolor[rgb]{0.73,0.40,0.53}{##1}}}
\expandafter\def\csname PY@tok@ss\endcsname{\def\PY@tc##1{\textcolor[rgb]{0.10,0.09,0.49}{##1}}}
\expandafter\def\csname PY@tok@sx\endcsname{\def\PY@tc##1{\textcolor[rgb]{0.00,0.50,0.00}{##1}}}
\expandafter\def\csname PY@tok@m\endcsname{\def\PY@tc##1{\textcolor[rgb]{0.40,0.40,0.40}{##1}}}
\expandafter\def\csname PY@tok@gh\endcsname{\let\PY@bf=\textbf\def\PY@tc##1{\textcolor[rgb]{0.00,0.00,0.50}{##1}}}
\expandafter\def\csname PY@tok@gu\endcsname{\let\PY@bf=\textbf\def\PY@tc##1{\textcolor[rgb]{0.50,0.00,0.50}{##1}}}
\expandafter\def\csname PY@tok@gd\endcsname{\def\PY@tc##1{\textcolor[rgb]{0.63,0.00,0.00}{##1}}}
\expandafter\def\csname PY@tok@gi\endcsname{\def\PY@tc##1{\textcolor[rgb]{0.00,0.63,0.00}{##1}}}
\expandafter\def\csname PY@tok@gr\endcsname{\def\PY@tc##1{\textcolor[rgb]{1.00,0.00,0.00}{##1}}}
\expandafter\def\csname PY@tok@ge\endcsname{\let\PY@it=\textit}
\expandafter\def\csname PY@tok@gs\endcsname{\let\PY@bf=\textbf}
\expandafter\def\csname PY@tok@gp\endcsname{\let\PY@bf=\textbf\def\PY@tc##1{\textcolor[rgb]{0.00,0.00,0.50}{##1}}}
\expandafter\def\csname PY@tok@go\endcsname{\def\PY@tc##1{\textcolor[rgb]{0.53,0.53,0.53}{##1}}}
\expandafter\def\csname PY@tok@gt\endcsname{\def\PY@tc##1{\textcolor[rgb]{0.00,0.27,0.87}{##1}}}
\expandafter\def\csname PY@tok@err\endcsname{\def\PY@bc##1{\setlength{\fboxsep}{0pt}\fcolorbox[rgb]{1.00,0.00,0.00}{1,1,1}{\strut ##1}}}
\expandafter\def\csname PY@tok@kc\endcsname{\let\PY@bf=\textbf\def\PY@tc##1{\textcolor[rgb]{0.00,0.50,0.00}{##1}}}
\expandafter\def\csname PY@tok@kd\endcsname{\let\PY@bf=\textbf\def\PY@tc##1{\textcolor[rgb]{0.00,0.50,0.00}{##1}}}
\expandafter\def\csname PY@tok@kn\endcsname{\let\PY@bf=\textbf\def\PY@tc##1{\textcolor[rgb]{0.00,0.50,0.00}{##1}}}
\expandafter\def\csname PY@tok@kr\endcsname{\let\PY@bf=\textbf\def\PY@tc##1{\textcolor[rgb]{0.00,0.50,0.00}{##1}}}
\expandafter\def\csname PY@tok@bp\endcsname{\def\PY@tc##1{\textcolor[rgb]{0.00,0.50,0.00}{##1}}}
\expandafter\def\csname PY@tok@fm\endcsname{\def\PY@tc##1{\textcolor[rgb]{0.00,0.00,1.00}{##1}}}
\expandafter\def\csname PY@tok@vc\endcsname{\def\PY@tc##1{\textcolor[rgb]{0.10,0.09,0.49}{##1}}}
\expandafter\def\csname PY@tok@vg\endcsname{\def\PY@tc##1{\textcolor[rgb]{0.10,0.09,0.49}{##1}}}
\expandafter\def\csname PY@tok@vi\endcsname{\def\PY@tc##1{\textcolor[rgb]{0.10,0.09,0.49}{##1}}}
\expandafter\def\csname PY@tok@vm\endcsname{\def\PY@tc##1{\textcolor[rgb]{0.10,0.09,0.49}{##1}}}
\expandafter\def\csname PY@tok@sa\endcsname{\def\PY@tc##1{\textcolor[rgb]{0.73,0.13,0.13}{##1}}}
\expandafter\def\csname PY@tok@sb\endcsname{\def\PY@tc##1{\textcolor[rgb]{0.73,0.13,0.13}{##1}}}
\expandafter\def\csname PY@tok@sc\endcsname{\def\PY@tc##1{\textcolor[rgb]{0.73,0.13,0.13}{##1}}}
\expandafter\def\csname PY@tok@dl\endcsname{\def\PY@tc##1{\textcolor[rgb]{0.73,0.13,0.13}{##1}}}
\expandafter\def\csname PY@tok@s2\endcsname{\def\PY@tc##1{\textcolor[rgb]{0.73,0.13,0.13}{##1}}}
\expandafter\def\csname PY@tok@sh\endcsname{\def\PY@tc##1{\textcolor[rgb]{0.73,0.13,0.13}{##1}}}
\expandafter\def\csname PY@tok@s1\endcsname{\def\PY@tc##1{\textcolor[rgb]{0.73,0.13,0.13}{##1}}}
\expandafter\def\csname PY@tok@mb\endcsname{\def\PY@tc##1{\textcolor[rgb]{0.40,0.40,0.40}{##1}}}
\expandafter\def\csname PY@tok@mf\endcsname{\def\PY@tc##1{\textcolor[rgb]{0.40,0.40,0.40}{##1}}}
\expandafter\def\csname PY@tok@mh\endcsname{\def\PY@tc##1{\textcolor[rgb]{0.40,0.40,0.40}{##1}}}
\expandafter\def\csname PY@tok@mi\endcsname{\def\PY@tc##1{\textcolor[rgb]{0.40,0.40,0.40}{##1}}}
\expandafter\def\csname PY@tok@il\endcsname{\def\PY@tc##1{\textcolor[rgb]{0.40,0.40,0.40}{##1}}}
\expandafter\def\csname PY@tok@mo\endcsname{\def\PY@tc##1{\textcolor[rgb]{0.40,0.40,0.40}{##1}}}
\expandafter\def\csname PY@tok@ch\endcsname{\let\PY@it=\textit\def\PY@tc##1{\textcolor[rgb]{0.25,0.50,0.50}{##1}}}
\expandafter\def\csname PY@tok@cm\endcsname{\let\PY@it=\textit\def\PY@tc##1{\textcolor[rgb]{0.25,0.50,0.50}{##1}}}
\expandafter\def\csname PY@tok@cpf\endcsname{\let\PY@it=\textit\def\PY@tc##1{\textcolor[rgb]{0.25,0.50,0.50}{##1}}}
\expandafter\def\csname PY@tok@c1\endcsname{\let\PY@it=\textit\def\PY@tc##1{\textcolor[rgb]{0.25,0.50,0.50}{##1}}}
\expandafter\def\csname PY@tok@cs\endcsname{\let\PY@it=\textit\def\PY@tc##1{\textcolor[rgb]{0.25,0.50,0.50}{##1}}}

\def\PYZbs{\char`\\}
\def\PYZus{\char`\_}
\def\PYZob{\char`\{}
\def\PYZcb{\char`\}}
\def\PYZca{\char`\^}
\def\PYZam{\char`\&}
\def\PYZlt{\char`\<}
\def\PYZgt{\char`\>}
\def\PYZsh{\char`\#}
\def\PYZpc{\char`\%}
\def\PYZdl{\char`\$}
\def\PYZhy{\char`\-}
\def\PYZsq{\char`\'}
\def\PYZdq{\char`\"}
\def\PYZti{\char`\~}
% for compatibility with earlier versions
\def\PYZat{@}
\def\PYZlb{[}
\def\PYZrb{]}
\makeatother


    % Exact colors from NB
    \definecolor{incolor}{rgb}{0.0, 0.0, 0.5}
    \definecolor{outcolor}{rgb}{0.545, 0.0, 0.0}



    
    % Prevent overflowing lines due to hard-to-break entities
    \sloppy 
    % Setup hyperref package
    \hypersetup{
      breaklinks=true,  % so long urls are correctly broken across lines
      colorlinks=true,
      urlcolor=urlcolor,
      linkcolor=linkcolor,
      citecolor=citecolor,
      }
    % Slightly bigger margins than the latex defaults
    
    \geometry{verbose,tmargin=1in,bmargin=1in,lmargin=1in,rmargin=1in}
    
    

    \begin{document}
    
    
    \maketitle
    
    

    
    \section{Homework 1: Causality and
Expressions}\label{homework-1-causality-and-expressions}

Please complete this notebook by filling in the cells provided. Before
you begin, execute the following cell to load the provided tests.

    \begin{Verbatim}[commandchars=\\\{\}]
{\color{incolor}In [{\color{incolor} }]:} \PY{c+c1}{\PYZsh{} Don\PYZsq{}t change this cell; just run it. }
        \PY{c+c1}{\PYZsh{} When you log\PYZhy{}in please hit return (not shift + return) after typing in your email}
        \PY{k+kn}{from} \PY{n+nn}{client}\PY{n+nn}{.}\PY{n+nn}{api}\PY{n+nn}{.}\PY{n+nn}{notebook} \PY{k}{import} \PY{n}{Notebook}
        \PY{n}{ok} \PY{o}{=} \PY{n}{Notebook}\PY{p}{(}\PY{l+s+s1}{\PYZsq{}}\PY{l+s+s1}{hw01.ok}\PY{l+s+s1}{\PYZsq{}}\PY{p}{)}
        \PY{c+c1}{\PYZsh{}ok.auth()}
\end{Verbatim}


    \textbf{Recommended Reading:} -
\href{http://www.inferentialthinking.com/chapters/01/what-is-data-science.html}{What
is Data Science} -
\href{http://www.inferentialthinking.com/chapters/02/causality-and-experiments.html}{Causality
and Experiments} -
\href{http://www.inferentialthinking.com/chapters/03/programming-in-python.html}{Programming
in Python}

For all problems that you must write explanations and sentences for, you
\textbf{must} provide your answer in the designated space. Moreover,
throughout this homework and all future ones, please be sure to not
re-assign variables throughout the notebook! For example, if you use
\texttt{max\_temperature} in your answer to one question, do not
reassign it later on. Otherwise, you will fail tests that you thought
you were passing previously!

Directly sharing answers is not okay, but discussing problems with the
course staff or with other students is encouraged. Refer to the policies
page to learn more about how to learn cooperatively.

You should start early so that you have time to get help if you're
stuck.

    Before continuing the assignment, select "Save and Checkpoint" in the
File menu and then execute the \texttt{submit} cell below. The result
will contain a link that you can use to check that your assignment has
been submitted successfully. If you submit more than once before the
deadline, we will only grade your final submission. If you mistakenly
submit the wrong one, you can head to okpy.org and flag the correct
version. There will be another \texttt{submit} cell at the end of the
assignment when you finish!

    \begin{Verbatim}[commandchars=\\\{\}]
{\color{incolor}In [{\color{incolor} }]:} \PY{n}{\PYZus{}} \PY{o}{=} \PY{n}{ok}\PY{o}{.}\PY{n}{submit}\PY{p}{(}\PY{p}{)}
\end{Verbatim}


    \subsection{1. Scary Arithmetic}\label{scary-arithmetic}

    An ad for ADT Security Systems says,

\begin{quote}
"When you go on vacation, burglars go to work {[}...{]} According to FBI
statistics, over 25\% of home burglaries occur between Memorial Day and
Labor Day."
\end{quote}

Do the data in the ad support the claim that burglars are more likely to
go to work during the time between Memorial Day and Labor Day? Please
explain your answer.

\textbf{Note:} You can assume that "over 25\%" means only slightly over.
Had it been much over, say closer to 30\%, then the marketers would have
said so.

    \emph{Write your answer here, replacing this text.}

    \subsection{2. Characters in Little
Women}\label{characters-in-little-women}

    In lecture, we counted the number of times that the literary characters
were named in each chapter of the classic book,
\href{https://www.inferentialthinking.com/chapters/01/3/1/literary-characters}{\emph{Little
Women}}. In computer science, the word "character" also refers to a
letter, digit, space, or punctuation mark; any single element of a text.
The following code generates a scatter plot in which each dot
corresponds to a chapter of \emph{Little Women}. The horizontal position
of a dot measures the number of periods in the chapter. The vertical
position measures the total number of characters.

    \begin{Verbatim}[commandchars=\\\{\}]
{\color{incolor}In [{\color{incolor} }]:} \PY{c+c1}{\PYZsh{} This cell contains code that hasn\PYZsq{}t yet been covered in the course,}
        \PY{c+c1}{\PYZsh{} but you should be able to interpret the scatter plot it generates.}
        
        \PY{k+kn}{from} \PY{n+nn}{datascience} \PY{k}{import} \PY{o}{*}
        \PY{k+kn}{from} \PY{n+nn}{urllib}\PY{n+nn}{.}\PY{n+nn}{request} \PY{k}{import} \PY{n}{urlopen}
        \PY{k+kn}{import} \PY{n+nn}{numpy} \PY{k}{as} \PY{n+nn}{np}
        \PY{o}{\PYZpc{}}\PY{k}{matplotlib} inline
        
        \PY{n}{little\PYZus{}women\PYZus{}url} \PY{o}{=} \PY{l+s+s1}{\PYZsq{}}\PY{l+s+s1}{https://www.inferentialthinking.com/data/little\PYZus{}women.txt}\PY{l+s+s1}{\PYZsq{}}
        \PY{n}{chapters} \PY{o}{=} \PY{n}{urlopen}\PY{p}{(}\PY{n}{little\PYZus{}women\PYZus{}url}\PY{p}{)}\PY{o}{.}\PY{n}{read}\PY{p}{(}\PY{p}{)}\PY{o}{.}\PY{n}{decode}\PY{p}{(}\PY{p}{)}\PY{o}{.}\PY{n}{split}\PY{p}{(}\PY{l+s+s1}{\PYZsq{}}\PY{l+s+s1}{CHAPTER }\PY{l+s+s1}{\PYZsq{}}\PY{p}{)}\PY{p}{[}\PY{l+m+mi}{1}\PY{p}{:}\PY{p}{]}
        \PY{n}{text} \PY{o}{=} \PY{n}{Table}\PY{p}{(}\PY{p}{)}\PY{o}{.}\PY{n}{with\PYZus{}column}\PY{p}{(}\PY{l+s+s1}{\PYZsq{}}\PY{l+s+s1}{Chapters}\PY{l+s+s1}{\PYZsq{}}\PY{p}{,} \PY{n}{chapters}\PY{p}{)}
        \PY{n}{Table}\PY{p}{(}\PY{p}{)}\PY{o}{.}\PY{n}{with\PYZus{}columns}\PY{p}{(}
            \PY{l+s+s1}{\PYZsq{}}\PY{l+s+s1}{Periods}\PY{l+s+s1}{\PYZsq{}}\PY{p}{,}    \PY{n}{np}\PY{o}{.}\PY{n}{char}\PY{o}{.}\PY{n}{count}\PY{p}{(}\PY{n}{chapters}\PY{p}{,} \PY{l+s+s1}{\PYZsq{}}\PY{l+s+s1}{.}\PY{l+s+s1}{\PYZsq{}}\PY{p}{)}\PY{p}{,}
            \PY{l+s+s1}{\PYZsq{}}\PY{l+s+s1}{Characters}\PY{l+s+s1}{\PYZsq{}}\PY{p}{,} \PY{n}{text}\PY{o}{.}\PY{n}{apply}\PY{p}{(}\PY{n+nb}{len}\PY{p}{,} \PY{l+m+mi}{0}\PY{p}{)}
            \PY{p}{)}\PY{o}{.}\PY{n}{scatter}\PY{p}{(}\PY{l+m+mi}{0}\PY{p}{)}
\end{Verbatim}


    \textbf{Question 1.} Around how many periods are there in the chapter
with the most characters? Assign either 1, 2, 3, 4, or 5 to the name
\texttt{characters\_q1} below.

\begin{enumerate}
\def\labelenumi{\arabic{enumi}.}
\tightlist
\item
  250
\item
  390
\item
  440
\item
  32,000
\item
  40,000
\end{enumerate}

    \begin{Verbatim}[commandchars=\\\{\}]
{\color{incolor}In [{\color{incolor} }]:} \PY{n}{characters\PYZus{}q1} \PY{o}{=} \PY{o}{.}\PY{o}{.}\PY{o}{.}
\end{Verbatim}


    \begin{Verbatim}[commandchars=\\\{\}]
{\color{incolor}In [{\color{incolor} }]:} \PY{n}{ok}\PY{o}{.}\PY{n}{grade}\PY{p}{(}\PY{l+s+s2}{\PYZdq{}}\PY{l+s+s2}{q2\PYZus{}1}\PY{l+s+s2}{\PYZdq{}}\PY{p}{)}\PY{p}{;}
\end{Verbatim}


    The test above checks that your answers are in the correct format.
\textbf{This test does not check that you answered correctly}, only that
you assigned a number successfully in each multiple-choice answer cell.

    \textbf{Question 2.} Which of the following chapters has the most
characters per period? Assign either 1, 2, or 3 to the name
\texttt{characters\_q2} below. 1. The chapter with about 60 periods 2.
The chapter with about 350 periods 3. The chapter with about 440 periods

    \begin{Verbatim}[commandchars=\\\{\}]
{\color{incolor}In [{\color{incolor} }]:} \PY{n}{characters\PYZus{}q2} \PY{o}{=} \PY{o}{.}\PY{o}{.}\PY{o}{.}
\end{Verbatim}


    \begin{Verbatim}[commandchars=\\\{\}]
{\color{incolor}In [{\color{incolor} }]:} \PY{n}{ok}\PY{o}{.}\PY{n}{grade}\PY{p}{(}\PY{l+s+s2}{\PYZdq{}}\PY{l+s+s2}{q2\PYZus{}2}\PY{l+s+s2}{\PYZdq{}}\PY{p}{)}\PY{p}{;}
\end{Verbatim}


    Again, the test above checks that your answers are in the correct
format, but not that you have answered correctly.

    To discover more interesting facts from this plot, read
\href{https://www.inferentialthinking.com/chapters/01/3/2/another-kind-of-character}{Section
1.3.2} of the textbook.

    \subsection{3. Names and Assignment
Statements}\label{names-and-assignment-statements}

    \textbf{Question 1.} When you run the following cell, Python produces a
cryptic error message.

    \begin{Verbatim}[commandchars=\\\{\}]
{\color{incolor}In [{\color{incolor} }]:} \PY{l+m+mi}{4} \PY{o}{=} \PY{l+m+mi}{2} \PY{o}{+} \PY{l+m+mi}{2}
\end{Verbatim}


    Choose the best explanation of what's wrong with the code, and then
assign 1, 2, 3, or 4 to \texttt{names\_q1} below to indicate your
answer.

\begin{enumerate}
\def\labelenumi{\arabic{enumi}.}
\item
  Python is smart and already knows \texttt{4\ =\ 2\ +\ 2}.
\item
  \texttt{4} is already a defined number, and it doesn't make sense to
  make a number be a name for something else. In Python,
  "\texttt{x\ =\ 2\ +\ 2}" means "assign \texttt{x} as the name for the
  value of \texttt{2\ +\ 2}."
\item
  It should be \texttt{2\ +\ 2\ =\ 4}.
\item
  I don't get an error message. This is a trick question.
\end{enumerate}

    \begin{Verbatim}[commandchars=\\\{\}]
{\color{incolor}In [{\color{incolor} }]:} \PY{n}{names\PYZus{}q1} \PY{o}{=} \PY{o}{.}\PY{o}{.}\PY{o}{.}
\end{Verbatim}


    \begin{Verbatim}[commandchars=\\\{\}]
{\color{incolor}In [{\color{incolor} }]:} \PY{n}{ok}\PY{o}{.}\PY{n}{grade}\PY{p}{(}\PY{l+s+s2}{\PYZdq{}}\PY{l+s+s2}{q3\PYZus{}1}\PY{l+s+s2}{\PYZdq{}}\PY{p}{)}\PY{p}{;}
\end{Verbatim}


    \textbf{Question 2.} When you run the following cell, Python will
produce another cryptic error message.

    \begin{Verbatim}[commandchars=\\\{\}]
{\color{incolor}In [{\color{incolor} }]:} \PY{n}{two} \PY{o}{=} \PY{l+m+mi}{3}
        \PY{n}{six} \PY{o}{=} \PY{n}{two} \PY{n}{plus} \PY{n}{two}
\end{Verbatim}


    Choose the best explanation of what's wrong with the code and assign 1,
2, 3, or 4 to \texttt{names\_q2} below to indicate your answer.

\begin{enumerate}
\def\labelenumi{\arabic{enumi}.}
\item
  The \texttt{plus} operation only applies to numbers, not the word
  "two".
\item
  The name "two" cannot be assigned to the number 3.
\item
  Two plus two is four, not six.
\item
  Python cannot interpret the name \texttt{two} followed directly by
  another name.
\end{enumerate}

    \begin{Verbatim}[commandchars=\\\{\}]
{\color{incolor}In [{\color{incolor} }]:} \PY{n}{names\PYZus{}q2} \PY{o}{=} \PY{o}{.}\PY{o}{.}\PY{o}{.}
\end{Verbatim}


    \begin{Verbatim}[commandchars=\\\{\}]
{\color{incolor}In [{\color{incolor} }]:} \PY{n}{ok}\PY{o}{.}\PY{n}{grade}\PY{p}{(}\PY{l+s+s2}{\PYZdq{}}\PY{l+s+s2}{q3\PYZus{}2}\PY{l+s+s2}{\PYZdq{}}\PY{p}{)}\PY{p}{;}
\end{Verbatim}


    \textbf{Question 3.} When you run the following cell, Python will, yet
again, produce another cryptic error message.

    \begin{Verbatim}[commandchars=\\\{\}]
{\color{incolor}In [{\color{incolor} }]:} \PY{n}{x} \PY{o}{=} \PY{n+nb}{print}\PY{p}{(}\PY{l+m+mi}{5}\PY{p}{)}
        \PY{n}{y} \PY{o}{=} \PY{n}{x} \PY{o}{+} \PY{l+m+mi}{2}
\end{Verbatim}


    Choose the best explanation of what's wrong with the code and assign 1,
2, or 3 to \texttt{names\_q3} below to indicate your answer.

\begin{enumerate}
\def\labelenumi{\arabic{enumi}.}
\item
  Python doesn't want \texttt{y} to be assigned.
\item
  The \texttt{print} operation is meant for displaying values to the
  programmer, not for assigning values!
\item
  What error message?
\end{enumerate}

    \begin{Verbatim}[commandchars=\\\{\}]
{\color{incolor}In [{\color{incolor} }]:} \PY{n}{names\PYZus{}q3} \PY{o}{=} \PY{o}{.}\PY{o}{.}\PY{o}{.}
\end{Verbatim}


    \begin{Verbatim}[commandchars=\\\{\}]
{\color{incolor}In [{\color{incolor} }]:} \PY{n}{ok}\PY{o}{.}\PY{n}{grade}\PY{p}{(}\PY{l+s+s2}{\PYZdq{}}\PY{l+s+s2}{q3\PYZus{}3}\PY{l+s+s2}{\PYZdq{}}\PY{p}{)}\PY{p}{;}
\end{Verbatim}


    \subsection{4. Job Opportunities \& Education in Rural
India}\label{job-opportunities-education-in-rural-india}

    A \href{http://www.nber.org/papers/w16021.pdf}{study} at UCLA
investigated factors that might result in greater attention to the
health and education of girls in rural India. One such factor is
information about job opportunities for women. The idea is that if
people know that educated women can get good jobs, they might take more
care of the health and education of girls in their families, as an
investment in the girls' future potential as earners. Without the
knowledge of job opportunities, the author hypothesizes that families do
not invest in women's well-being.

The study focused on 160 villages outside the capital of India, all with
little access to information about call centers and similar
organizations that offer job opportunities to women. In 80 of the
villages chosen at random, recruiters visited the village, described the
opportunities, recruited women who had some English language proficiency
and experience with computers, and provided ongoing support free of
charge for three years. In the other 80 villages, no recruiters visited
and no other intervention was made.

At the end of the study period, the researchers recorded data about the
school attendance and health of the children in the villages.

    \textbf{Question 1.} Which statement best describes the \emph{treatment}
and \emph{control} groups for this study? Assign either 1, 2, or 3 to
the name \texttt{jobs\_q1} below.

\begin{enumerate}
\def\labelenumi{\arabic{enumi}.}
\item
  The treatment group was the 80 villages visited by recruiters, and the
  control group was the other 80 villages with no intervention.
\item
  The treatment group was the 160 villages selected, and the control
  group was the rest of the villages outside the capital of India.
\item
  There is no clear notion of \emph{treatment} and \emph{control} group
  in this study.
\end{enumerate}

    \begin{Verbatim}[commandchars=\\\{\}]
{\color{incolor}In [{\color{incolor} }]:} \PY{n}{jobs\PYZus{}q1} \PY{o}{=} \PY{o}{.}\PY{o}{.}\PY{o}{.}
\end{Verbatim}


    \begin{Verbatim}[commandchars=\\\{\}]
{\color{incolor}In [{\color{incolor} }]:} \PY{n}{ok}\PY{o}{.}\PY{n}{grade}\PY{p}{(}\PY{l+s+s2}{\PYZdq{}}\PY{l+s+s2}{q4\PYZus{}1}\PY{l+s+s2}{\PYZdq{}}\PY{p}{)}\PY{p}{;}
\end{Verbatim}


    \textbf{Question 2.} Was this an observational study or a randomized
controlled experiment? Assign either 1, 2, or 3 to the name
\texttt{jobs\_q2} below.

\begin{enumerate}
\def\labelenumi{\arabic{enumi}.}
\item
  This was an observational study.
\item
  This was a randomized controlled experiment.
\item
  This was a randomized observational study.
\end{enumerate}

    \begin{Verbatim}[commandchars=\\\{\}]
{\color{incolor}In [{\color{incolor} }]:} \PY{n}{jobs\PYZus{}q2} \PY{o}{=} \PY{o}{.}\PY{o}{.}\PY{o}{.}
\end{Verbatim}


    \begin{Verbatim}[commandchars=\\\{\}]
{\color{incolor}In [{\color{incolor} }]:} \PY{n}{ok}\PY{o}{.}\PY{n}{grade}\PY{p}{(}\PY{l+s+s2}{\PYZdq{}}\PY{l+s+s2}{q4\PYZus{}2}\PY{l+s+s2}{\PYZdq{}}\PY{p}{)}\PY{p}{;}
\end{Verbatim}


    \textbf{Question 3.} The study reported, ``Girls aged 5-15 in villages
that received the recruiting services were 3 to 5 percentage points more
likely to be in school and experienced an increase in Body Mass Index,
reflecting greater nutrition and/or medical care. However, there was no
net gain in height. For boys, there was no change in any of these
measures.'' Why do you think the author points out the lack of change in
the boys?

\emph{Hint:} Remember the original hypothesis. The author believes that
educating women in job opportunities will cause families to invest more
in the women's well-being.

    \emph{Write your answer here, replacing this text.}

    \subsection{5. Differences between
Majors}\label{differences-between-majors}

    Berkeley's Office of Planning and Analysis provides data on numerous
aspects of the campus. Adapted from the OPA website, the table below
displays the numbers of degree recipients in three majors in the
academic years 2008-2009 and 2017-2018.

\begin{longtable}[]{@{}lll@{}}
\toprule
Major & 2008-2009 & 2017-2018\tabularnewline
\midrule
\endhead
Gender and Women's Studies & 17 & 28\tabularnewline
Linguistics & 49 & 67\tabularnewline
Rhetoric & 113 & 56\tabularnewline
\bottomrule
\end{longtable}

    \textbf{Question 1.} Suppose you want to find the \textbf{biggest}
absolute difference between the numbers of degree recipients in the two
years, among the three majors.

In the cell below, compute this value and call it
\texttt{biggest\_change}. Use a single expression (a single line of
code) to compute the answer. Let Python perform all the arithmetic (like
subtracting 49 from 67) rather than simplifying the expression yourself.
The built-in abs function takes a numerical input and returns the
absolute value.

    \begin{Verbatim}[commandchars=\\\{\}]
{\color{incolor}In [{\color{incolor} }]:} \PY{n}{biggest\PYZus{}change} \PY{o}{=} \PY{o}{.}\PY{o}{.}\PY{o}{.}
        \PY{n}{biggest\PYZus{}change}
\end{Verbatim}


    \begin{Verbatim}[commandchars=\\\{\}]
{\color{incolor}In [{\color{incolor} }]:} \PY{n}{ok}\PY{o}{.}\PY{n}{grade}\PY{p}{(}\PY{l+s+s2}{\PYZdq{}}\PY{l+s+s2}{q5\PYZus{}1}\PY{l+s+s2}{\PYZdq{}}\PY{p}{)}\PY{p}{;}
\end{Verbatim}


    Use the cell above to test for formatting (in this case, that
dissimilarity is a number)

    \textbf{Question 2.} Which of the three majors had the \textbf{smallest}
absolute difference? Assign \texttt{smallest\_change\_major} to 1, 2, or
3 where each number corresponds to the following major:

1: Gender and Women's Studies\\
2: Linguistics\\
3: Rhetoric

Choose the number that corresponds to the major with the smallest
absolute difference.

You should be able to answer by rough mental arithmetic, without having
to calculate the exact value for each major.

    \begin{Verbatim}[commandchars=\\\{\}]
{\color{incolor}In [{\color{incolor} }]:} \PY{n}{smallest\PYZus{}change\PYZus{}major} \PY{o}{=} \PY{o}{.}\PY{o}{.}\PY{o}{.}
        \PY{n}{smallest\PYZus{}change\PYZus{}major}
\end{Verbatim}


    \begin{Verbatim}[commandchars=\\\{\}]
{\color{incolor}In [{\color{incolor} }]:} \PY{n}{ok}\PY{o}{.}\PY{n}{grade}\PY{p}{(}\PY{l+s+s2}{\PYZdq{}}\PY{l+s+s2}{q5\PYZus{}2}\PY{l+s+s2}{\PYZdq{}}\PY{p}{)}\PY{p}{;}
\end{Verbatim}


    \textbf{Question 3.} For each major, define the ``relative change'' to
be the following:
\(\large{\frac{\text{absolute difference}}{\text{value in 2008-2009}} * 100}\)

Fill in the code below such that \texttt{gws\_relative\_change},
\texttt{linguistics\_relative\_change} and
\texttt{rhetoric\_relative\_change} are assigned to the relative changes
for their respective majors.

    \begin{Verbatim}[commandchars=\\\{\}]
{\color{incolor}In [{\color{incolor} }]:} \PY{n}{gws\PYZus{}relative\PYZus{}change} \PY{o}{=} \PY{p}{(}\PY{n+nb}{abs}\PY{p}{(}\PY{o}{.}\PY{o}{.}\PY{o}{.}\PY{p}{)} \PY{o}{/} \PY{l+m+mi}{17}\PY{p}{)} \PY{o}{*} \PY{l+m+mi}{100}
        \PY{n}{linguistics\PYZus{}relative\PYZus{}change} \PY{o}{=} \PY{o}{.}\PY{o}{.}\PY{o}{.}
        \PY{n}{rhetoric\PYZus{}relative\PYZus{}change} \PY{o}{=} \PY{o}{.}\PY{o}{.}\PY{o}{.}
        \PY{n}{gws\PYZus{}relative\PYZus{}change}\PY{p}{,} \PY{n}{linguistics\PYZus{}relative\PYZus{}change}\PY{p}{,} \PY{n}{rhetoric\PYZus{}relative\PYZus{}change}
\end{Verbatim}


    \begin{Verbatim}[commandchars=\\\{\}]
{\color{incolor}In [{\color{incolor} }]:} \PY{n}{ok}\PY{o}{.}\PY{n}{grade}\PY{p}{(}\PY{l+s+s2}{\PYZdq{}}\PY{l+s+s2}{q5\PYZus{}3}\PY{l+s+s2}{\PYZdq{}}\PY{p}{)}\PY{p}{;}
\end{Verbatim}


    \textbf{Question 4.} Assign \texttt{biggest\_rel\_change\_major} to 1,
2, or 3 where each number corresponds to to the following:

1: Gender and Women's Studies\\
2: Linguistics\\
3: Rhetoric

Choose the number that corresponds to the major with the biggest
relative change.

    \begin{Verbatim}[commandchars=\\\{\}]
{\color{incolor}In [{\color{incolor} }]:} \PY{c+c1}{\PYZsh{} Assign biggest\PYZus{}rel\PYZus{}change\PYZus{}major to the number corresponding to the major with the biggest relative change.}
        \PY{n}{biggest\PYZus{}rel\PYZus{}change\PYZus{}major} \PY{o}{=} \PY{o}{.}\PY{o}{.}\PY{o}{.}
        \PY{n}{biggest\PYZus{}rel\PYZus{}change\PYZus{}major}
\end{Verbatim}


    \begin{Verbatim}[commandchars=\\\{\}]
{\color{incolor}In [{\color{incolor} }]:} \PY{n}{ok}\PY{o}{.}\PY{n}{grade}\PY{p}{(}\PY{l+s+s2}{\PYZdq{}}\PY{l+s+s2}{q5\PYZus{}4}\PY{l+s+s2}{\PYZdq{}}\PY{p}{)}\PY{p}{;}
\end{Verbatim}


    \subsection{6. Nearsightedness Study}\label{nearsightedness-study}

    Myopia, or nearsightedness, results from a number of genetic and
environmental factors. In 1999, Quinn et al studied the relation between
myopia and ambient lighting at night (for example, from nightlights or
room lights) during childhood.

    \textbf{Question 1.} The data were gathered by the following procedure,
reported in the study. ``Between January and June 1998, parents of
children aged 2-16 years {[}...{]} that were seen as outpatients in a
university pediatric ophthalmology clinic completed a questionnaire on
the child's light exposure both at present and before the age of 2
years.'' Was this study observational, or was it a controlled
experiment? Explain.

    \emph{Write your answer here, replacing this text.}

    \textbf{Question 2.} The study found that of the children who slept with
a room light on before the age of 2, 55\% were myopic. Of the children
who slept with a night light on before the age of 2, 34\% were myopic.
Of the children who slept in the dark before the age of 2, 10\% were
myopic. The study concluded that, "The prevalence of myopia {[}...{]}
during childhood was strongly associated with ambient light exposure
during sleep at night in the first two years after birth."

Do the data support this statement? You may interpret ``strongly'' in
any reasonable qualitative way.

    \emph{Write your answer here, replacing this text.}

    \textbf{Question 3.} On May 13, 1999, CNN reported the results of this
study under the headline, ``Night light may lead to nearsightedness.''
Does the conclusion of the study claim that night light causes
nearsightedness?

    \emph{Write your answer here, replacing this text.}

    \textbf{Question 4.} The final paragraph of the CNN report said that
``several eye specialists'' had pointed out that the study should have
accounted for heredity.

Myopia is passed down from parents to children. Myopic parents are more
likely to have myopic children, and may also be more likely to leave
lights on habitually (since they have poor vision). In what way does the
knowledge of this possible genetic link affect how we interpret the data
from the study?

    \emph{Write your answer here, replacing this text.}

    \subsection{7. Studying the Survivors}\label{studying-the-survivors}

    The Reverend Henry Whitehead was skeptical of John Snow's conclusion
about the Broad Street pump. After the Broad Street cholera epidemic
ended, Whitehead set about trying to prove Snow wrong. (The history of
the event is detailed
\href{http://www.ncbi.nlm.nih.gov/pmc/articles/PMC1034367/pdf/medhist00183-0026.pdf}{here}.)

He realized that Snow had focused his analysis almost entirely on those
who had died. Whitehead, therefore, investigated the drinking habits of
people in the Broad Street area who had not died in the outbreak.

What is the main reason it was important to study this group?

\begin{enumerate}
\def\labelenumi{\arabic{enumi})}
\item
  If Whitehead had found that many people had drunk water from the Broad
  Street pump and not caught cholera, that would have been evidence
  against Snow's hypothesis.
\item
  Survivors could provide additional information about what else could
  have caused the cholera, potentially unearthing another cause.
\item
  Through considering the survivors, Whitehead could have identified a
  cure for cholera.
\end{enumerate}

    \begin{Verbatim}[commandchars=\\\{\}]
{\color{incolor}In [{\color{incolor} }]:} \PY{c+c1}{\PYZsh{} Assign survivor\PYZus{}answer to 1, 2, or 3}
        \PY{n}{survivor\PYZus{}answer} \PY{o}{=} \PY{o}{.}\PY{o}{.}\PY{o}{.}
\end{Verbatim}


    \begin{Verbatim}[commandchars=\\\{\}]
{\color{incolor}In [{\color{incolor} }]:} \PY{n}{ok}\PY{o}{.}\PY{n}{grade}\PY{p}{(}\PY{l+s+s2}{\PYZdq{}}\PY{l+s+s2}{q7\PYZus{}1}\PY{l+s+s2}{\PYZdq{}}\PY{p}{)}\PY{p}{;}
\end{Verbatim}


    \textbf{Note:} Whitehead ended up finding further proof that the Broad
Street pump played the central role in spreading the disease to the
people who lived near it. Eventually, he became one of Snow's greatest
defenders.

    \subsection{8. Submission}\label{submission}

    Once you're finished, select "Save and Checkpoint" in the File menu and
then execute the \texttt{submit} cell below.

    \begin{Verbatim}[commandchars=\\\{\}]
{\color{incolor}In [{\color{incolor} }]:} \PY{n}{\PYZus{}} \PY{o}{=} \PY{n}{ok}\PY{o}{.}\PY{n}{submit}\PY{p}{(}\PY{p}{)}
\end{Verbatim}


    \begin{Verbatim}[commandchars=\\\{\}]
{\color{incolor}In [{\color{incolor} }]:} \PY{c+c1}{\PYZsh{} For your convenience, you can run this cell to run all the tests at once!}
        \PY{k+kn}{import} \PY{n+nn}{os}
        \PY{n+nb}{print}\PY{p}{(}\PY{l+s+s2}{\PYZdq{}}\PY{l+s+s2}{Running all tests...}\PY{l+s+s2}{\PYZdq{}}\PY{p}{)}
        \PY{n}{\PYZus{}} \PY{o}{=} \PY{p}{[}\PY{n}{ok}\PY{o}{.}\PY{n}{grade}\PY{p}{(}\PY{n}{q}\PY{p}{[}\PY{p}{:}\PY{o}{\PYZhy{}}\PY{l+m+mi}{3}\PY{p}{]}\PY{p}{)} \PY{k}{for} \PY{n}{q} \PY{o+ow}{in} \PY{n}{os}\PY{o}{.}\PY{n}{listdir}\PY{p}{(}\PY{l+s+s2}{\PYZdq{}}\PY{l+s+s2}{tests}\PY{l+s+s2}{\PYZdq{}}\PY{p}{)} \PY{k}{if} \PY{n}{q}\PY{o}{.}\PY{n}{startswith}\PY{p}{(}\PY{l+s+s1}{\PYZsq{}}\PY{l+s+s1}{q}\PY{l+s+s1}{\PYZsq{}}\PY{p}{)} \PY{o+ow}{and} \PY{n+nb}{len}\PY{p}{(}\PY{n}{q}\PY{p}{)} \PY{o}{\PYZlt{}}\PY{o}{=} \PY{l+m+mi}{10}\PY{p}{]}
        \PY{n+nb}{print}\PY{p}{(}\PY{l+s+s2}{\PYZdq{}}\PY{l+s+s2}{Finished running all tests.}\PY{l+s+s2}{\PYZdq{}}\PY{p}{)}
\end{Verbatim}



    % Add a bibliography block to the postdoc
    
    
    
    \end{document}
